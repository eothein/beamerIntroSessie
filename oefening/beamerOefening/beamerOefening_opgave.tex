

%%%%%%%%%%%%%%%%%%%%%%%%%%%%%%%%%%%%%%%%%
% Beamer Presentation
% LaTeX Template
% Version 1.0 (10/11/12)
%
% This template has been downloaded from:
% http://www.LaTeXTemplates.com
%
% License:
% CC BY-NC-SA 3.0 (http://creativecommons.org/licenses/by-nc-sa/3.0/)
%
%%%%%%%%%%%%%%%%%%%%%%%%%%%%%%%%%%%%%%%%%

%----------------------------------------------------------------------------------------
%	PACKAGES AND THEMES
%----------------------------------------------------------------------------------------

\documentclass{beamer}
%\setbeameroption{show notes}
\mode<presentation> {

% The Beamer class comes with a number of default slide themes
% which change the colors and layouts of slides. Below this is a list
% of all the themes, uncomment each in turn to see what they look like.

%\usetheme{default}
%\usetheme{AnnArbor}
%\usetheme{Antibes}
%\usetheme{Bergen}
%\usetheme{Berkeley}
%\usetheme{Berlin}
%\usetheme{Boadilla}
%\usetheme{CambridgeUS}
%\usetheme{Copenhagen}
%\usetheme{Darmstadt}
%\usetheme{Dresden}
%\usetheme{Frankfurt}
%\usetheme{Goettingen}
%\usetheme{Hannover}
%\usetheme{Ilmenau}
%\usetheme{JuanLesPins}
%\usetheme{Luebeck}
%\usetheme{Madrid}
%\usetheme{Malmoe}
%\usetheme{Marburg}
%\usetheme{Montpellier}
%\usetheme{PaloAlto}
%\usetheme{Pittsburgh}
%\usetheme{Rochester}
%\usetheme{Singapore}
%\usetheme{Szeged}
%\usetheme{Warsaw}

% As well as themes, the Beamer class has a number of color themes
% for any slide theme. Uncomment each of these in turn to see how it
% changes the colors of your current slide theme.

%\usecolortheme{albatross}
%\usecolortheme{beaver}
%\usecolortheme{beetle}
%\usecolortheme{crane}
%\usecolortheme{dolphin}
%\usecolortheme{dove}
%\usecolortheme{fly}
%\usecolortheme{lily}
%\usecolortheme{orchid}
%\usecolortheme{rose}
%\usecolortheme{seagull}
%\usecolortheme{seahorse}
%\usecolortheme{whale}
%\usecolortheme{wolverine}

%\setbeamertemplate{footline} % To remove the footer line in all slides uncomment this line
%\setbeamertemplate{footline}[page number] % To replace the footer line in all slides with a simple slide count uncomment this line

%\setbeamertemplate{navigation symbols}{} % To remove the navigation symbols from the bottom of all slides uncomment this line
}

\usepackage{graphicx} % Allows including images
\usepackage{booktabs} % Allows the use of \toprule, \midrule and \bottomrule in tables
\usepackage{listings, color}
\usepackage{lstautogobble}
\usepackage[dutch,english]{babel}

\graphicspath{ {./images/} }

%----------------------------------------------------------------------------------------
%	TITLE PAGE
%----------------------------------------------------------------------------------------

\title[Korte titel]{Lange Titel} % The short title appears at the bottom of every slide, the full title is only on the title page

\author{Uw naam} % Your name
\institute[HoGent - neem ik aan] % Your institution as it will appear on the bottom of every slide, may be shorthand to save space
{
Hogeschool Gent \\ % Your institution for the title page
\medskip
\textit{email adres} % Your email address
}
\date{\today} % Date, can be changed to a custom date

\begin{document}

\begin{frame}
\titlepage % Print the title page as the first slide
\end{frame}



%----------------------------------------------------------------------------------------
%	PRESENTATION SLIDES
%----------------------------------------------------------------------------------------

% OPGAVE 1
% Het maken een overzichtsframe
%


%OPGAVE 2
% Maken een gewone slide met wat simpele tekst


%OPGAVE 3
% Maak wat tekst en probeer verschillende stijlen toe te passen. Bold, italic, alert tekst, 

% OPGAVE 4
% Voeg een slide toe met wat java code (met lstlisting). Probeer de stijl van de code
% wat aan te passen.
% Java code:
%	    @Override
    %public boolean onCreateOptionsMenu(Menu menu) {
        %// Inflate the menu; this adds items to the action bar if it is present.
        %getMenuInflater().inflate(R.menu.to_do_list, menu);
        %return true;
    %}
% Tip: http://en.wikibooks.org/wiki/LaTeX/Source_Code_Listings


%OPGAVE 5
% Conditioneel builden: afhankelijk van een variabele pas je de inhoud van de slide aan
% Zoek hiervoor eens de if statement in latex op

%Opgave 6
% Invoegen van figuur (figure.jpg) centraal, 60 % van breedte

%OPGAVE 7
%Figuur die ganse breedte inneemt, zonder fransjes van beamer (baby.jpg)


%Opgave 8 
% Maak een genummerde lijst van volgende zaken. Zorg ervoor dat de teskt gradueel komt (tip: pause) TITEL: Things you'll never 	hear a man say, 1. "Does this hunter outfit make me look fat?" 2. "My wife never listens to me" 3. "I'll have a light salad without dressing and a diet cola" 5. "Looks like it is time to buy some new underwear"

%Opgave 9
% maak een slide bestaande uit twee kolommen waar in de eerste kolom voordelen van lachen staan, in de tweede kolom enkele nadelen
% Tip: http://tex.stackexchange.com/questions/32931/multiple-columns-with-images-and-wrapped-text-in-beamer.


\end{document} 
