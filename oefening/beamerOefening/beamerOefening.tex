

%%%%%%%%%%%%%%%%%%%%%%%%%%%%%%%%%%%%%%%%%
% Beamer Presentation
% LaTeX Template
% Version 1.0 (10/11/12)
%
% This template has been downloaded from:
% http://www.LaTeXTemplates.com
%
% License:
% CC BY-NC-SA 3.0 (http://creativecommons.org/licenses/by-nc-sa/3.0/)
%
%%%%%%%%%%%%%%%%%%%%%%%%%%%%%%%%%%%%%%%%%

%----------------------------------------------------------------------------------------
%	PACKAGES AND THEMES
%----------------------------------------------------------------------------------------

\documentclass{beamer}
%\setbeameroption{show notes}
\mode<presentation> {

% The Beamer class comes with a number of default slide themes
% which change the colors and layouts of slides. Below this is a list
% of all the themes, uncomment each in turn to see what they look like.

%\usetheme{default}
%\usetheme{AnnArbor}
%\usetheme{Antibes}
%\usetheme{Bergen}
%\usetheme{Berkeley}
%\usetheme{Berlin}
%\usetheme{Boadilla}
%\usetheme{CambridgeUS}
%\usetheme{Copenhagen}
%\usetheme{Darmstadt}
%\usetheme{Dresden}
%\usetheme{Frankfurt}
%\usetheme{Goettingen}
%\usetheme{Hannover}
%\usetheme{Ilmenau}
%\usetheme{JuanLesPins}
%\usetheme{Luebeck}
\usetheme{Madrid}
%\usetheme{Malmoe}
%\usetheme{Marburg}
%\usetheme{Montpellier}
%\usetheme{PaloAlto}
%\usetheme{Pittsburgh}
%\usetheme{Rochester}
%\usetheme{Singapore}
%\usetheme{Szeged}
%\usetheme{Warsaw}

% As well as themes, the Beamer class has a number of color themes
% for any slide theme. Uncomment each of these in turn to see how it
% changes the colors of your current slide theme.

%\usecolortheme{albatross}
\usecolortheme{beaver}
%\usecolortheme{beetle}
%\usecolortheme{crane}
%\usecolortheme{dolphin}
%\usecolortheme{dove}
%\usecolortheme{fly}
%\usecolortheme{lily}
%\usecolortheme{orchid}
%\usecolortheme{rose}
%\usecolortheme{seagull}
%\usecolortheme{seahorse}
%\usecolortheme{whale}
%\usecolortheme{wolverine}

%\setbeamertemplate{footline} % To remove the footer line in all slides uncomment this line
%\setbeamertemplate{footline}[page number] % To replace the footer line in all slides with a simple slide count uncomment this line

%\setbeamertemplate{navigation symbols}{} % To remove the navigation symbols from the bottom of all slides uncomment this line
}

\usepackage{graphicx} % Allows including images
\usepackage{listings, color}
\usepackage{lstautogobble}
\usepackage[dutch,english]{babel}
\usepackage{color}

\definecolor{mygreen}{rgb}{0,0.6,0}
\definecolor{mygray}{rgb}{0.5,0.5,0.5}
\definecolor{mymauve}{rgb}{0.58,0,0.82}

\newif\ifanswers
%\answerstrue

\graphicspath{ {./images/} }

%----------------------------------------------------------------------------------------
%	TITLE PAGE
%----------------------------------------------------------------------------------------

\title[Korte titel]{Lange Titel} % The short title appears at the bottom of every slide, the full title is only on the title page

\author{Uw naam} % Your name
\institute[HoGent - neem ik aan] % Your institution as it will appear on the bottom of every slide, may be shorthand to save space
{
Hogeschool Gent \\ % Your institution for the title page
\medskip
\textit{email adres} % Your email address
}
\date{\today} % Date, can be changed to a custom date

\begin{document}

\begin{frame}
\titlepage % Print the title page as the first slide
\end{frame}



\lstset{ %
  backgroundcolor=\color{white},   % choose the background color; you must add \usepackage{color} or \usepackage{xcolor}
  basicstyle=\footnotesize,        % the size of the fonts that are used for the code
  breakatwhitespace=false,         % sets if automatic breaks should only happen at whitespace
  breaklines=true,                 % sets automatic line breaking
  captionpos=b,                    % sets the caption-position to bottom
  commentstyle=\color{mygreen},    % comment style
  deletekeywords={...},            % if you want to delete keywords from the given language
  escapeinside={\%*}{*)},          % if you want to add LaTeX within your code
  extendedchars=true,              % lets you use non-ASCII characters; for 8-bits encodings only, does not work with UTF-8
  frame=single,                    % adds a frame around the code
  keepspaces=true,                 % keeps spaces in text, useful for keeping indentation of code (possibly needs columns=flexible)
  keywordstyle=\color{blue},       % keyword style
  language=Octave,                 % the language of the code
  morekeywords={*,...},            % if you want to add more keywords to the set
  numbers=left,                    % where to put the line-numbers; possible values are (none, left, right)
  numbersep=5pt,                   % how far the line-numbers are from the code
  numberstyle=\tiny\color{mygray}, % the style that is used for the line-numbers
  rulecolor=\color{black},         % if not set, the frame-color may be changed on line-breaks within not-black text (e.g. comments (green here))
  showspaces=false,                % show spaces everywhere adding particular underscores; it overrides 'showstringspaces'
  showstringspaces=false,          % underline spaces within strings only
  showtabs=false,                  % show tabs within strings adding particular underscores
  stepnumber=2,                    % the step between two line-numbers. If it's 1, each line will be numbered
  stringstyle=\color{mymauve},     % string literal style
  tabsize=2,                       % sets default tabsize to 2 spaces
  title=\lstname                   % show the filename of files included with \lstinputlisting; also try caption instead of title
}


%----------------------------------------------------------------------------------------
%	PRESENTATION SLIDES
%----------------------------------------------------------------------------------------



% OPGAVE 1
% Het maken een overzichtsframe
%

\section{Opgave 1}

\begin{frame}
	\frametitle{Overview}
	\tableofcontents[part=0,pausesections]
\end{frame}

\section{Opgave 2}
%OPGAVE 2
% Maken van een overzichtslide
\begin{frame}
	\frametitle{Hello World}
		Dit is een test slide
\end{frame}

\section{Opgave 3}
%OPGAVE 3
% Maak wat tekst en probeer verschillende stijlen toe te passen. Bold, italic, alert tekst, 
\begin{frame}
	\frametitle{Tekstopmaa}
	Dit is een \textbf{dikke tekst} en dit is \textit{italic} en dit is \alert{een alert tekst.}
\end{frame}

\section{Opgave 4}
% OPGAVE 4
% invoegen van java code
% Tip: http://en.wikibooks.org/wiki/LaTeX/Source_Code_Listings
\begin{frame}[fragile]
	\frametitle{Java code}
	\begin{lstlisting}[language=java, ]
    @Override
    public boolean onCreateOptionsMenu(Menu menu) {
        // Inflate the menu; this adds items to the action bar if it is present.
        getMenuInflater().inflate(R.menu.to_do_list, menu);
        return true;
    }
	\end{lstlisting}
\end{frame}

\section{Opgave 5}
%OPGAVE 5
% Conditioneel builden: afhankelijk van een variabele pas je de inhoud van de slide aan
\begin{frame}
	\frametitle{Condtioneel}
		Dit is standaard op de slide
	\ifanswers
		Dit is enkel als answers op true staat
	\fi
\end{frame}

\section{Opgave 6}
%Opgave 6
% Invoegen van figuur centraal, 60 % van breedte
\begin{frame}
	\frametitle{Figuur 1}
	
	\begin{figure}
		\centering
			\includegraphics[width=0.6\textwidth]{figure.jpg}
		\caption{Dit is een figuur}
		\label{fig:figure}
	\end{figure}
	
\end{frame}

\section{Opgave 7}
%OPGAVE 7
%Figuur die ganse breedte inneemt
\begin{frame}[plain]
	
	\begin{figure}
		\centering
			\includegraphics[width=1.00\textwidth]{baby.jpg}
		\label{fig:baby}
	\end{figure}
	
\end{frame}

\section{Opgave 8}
%Opgave 8 
% Maak een genummerde lijst van volgende zaken. Zorg ervoor dat de teskt gradueel komt (tip: pause) TITEL: Things you'll never 	hear a man say, 1. "Does this hunter outfit make me look fat?" 2. "My wife never listens to me" 3. "I'll have a light salad without dressing and a diet cola" 5. "Looks like it is time to buy some new underwear"

\begin{frame}
	\frametitle{Things you'll never 	hear a man say}
	
	\begin{enumerate}
		\item ``Does this hunter outfit make me look fat?'' \pause
		\item ``My wife never listens to me'' \pause
		\item ``I'll have a light salad without dressing and a diet cola'' \pause
		\item ``Looks like it is time to buy some new underwear'' \pause
	\end{enumerate}
\end{frame} 


\section{Opgave 9}
%Opgave 9
% maak een slide bestaande uit twee kolommen waar in de eerste kolom voordelen van lachen staan, in de tweede kolom enkele nadelen
% Tip: http://tex.stackexchange.com/questions/32931/multiple-columns-with-images-and-wrapped-text-in-beamer.

\begin{frame}
  \frametitle{This is a frame title}
  \begin{columns}[onlytextwidth]
    \begin{column}{0.5\textwidth}
      \centering
			\textbf{Voordelen van lachen}
					
					\begin{enumerate}
						\item Laughter causes deep muscle relaxation
						\item Laughter reduces stress
						\item Laughter has health benefits. 
						\item Laughter is a universal language.
						\item Laughter promotes communication.
					\end{enumerate}
    \end{column}
    \begin{column}{0.5\textwidth}
		\textbf{Nadelen van lachen}
      \centering
					\begin{enumerate}
						\item Peeing your pants 
						\item Dislocated jaw
						\item Headaches
						\item Emphysema
						\item Inhalation of foreign bodies (e.g. gum drops, popcorn, peppermint bark)
					\end{enumerate}
    \end{column}
​  \end{columns}
\end{frame}

\end{document} 
